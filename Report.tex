% Created 2023-03-24 Fri 17:19
% Intended LaTeX compiler: pdflatex
\documentclass[twocolumn]{article}
\usepackage[utf8]{inputenc}
\usepackage[T1]{fontenc}
\usepackage{graphicx}
\usepackage{longtable}
\usepackage{wrapfig}
\usepackage{rotating}
\usepackage[normalem]{ulem}
\usepackage{amsmath}
\usepackage{amssymb}
\usepackage{capt-of}
\usepackage{hyperref}
\usepackage{balance}
\usepackage{graphics}
\usepackage{txfonts}
\usepackage{times}
\usepackage{color}
\usepackage{textcomp}
\usepackage{booktabs}
\usepackage{todonotes}
\usepackage{float}
\usepackage{url}
\usepackage{titling}
\usepackage[pdftex]{hyperref}
\usepackage[british]{babel}
\usepackage{sectsty}
\sectionfont{\Large}
\subsectionfont{\large}
\subsubsectionfont{\large}
\paragraphfont{\normalsize}
\setlength{\parindent}{0em}
\setlength{\parskip}{1em}
\setlength{\columnsep}{2em}
\setlength{\droptitle}{-10em}
\makeatletter
\def\url@leostyle{%
\@ifundefined{selectfont}{\def\UrlFont{\sf}}{\def\UrlFont{\small\bf\ttfamily}}}
\makeatother
\urlstyle{leo}
\author{Struan Robertson \\ BSc (Hons) Applied Computing}
\date{May 2023}
\title{Terrain Model Processing with Machine Learning}
\hypersetup{
 pdfauthor={Struan Robertson \\ BSc (Hons) Applied Computing},
 pdftitle={Terrain Model Processing with Machine Learning},
 pdfkeywords={},
 pdfsubject={},
 pdfcreator={Emacs 28.2 (Org mode 9.6.1)}, 
 pdflang={English}}
\usepackage{biblatex}
\addbibresource{/home/struan/Dissertation/Report/library.bib}
\addbibresource{/home/struan/Sync/library.bib}
\begin{document}

\maketitle
\begin{abstract}

With the launch of the lunar orbiter laser altimeter (LOLA) on NASA's lunar reconnaissance orbiter (LRO), a large amount of high-resolution digital elevation maps (DEMs) have been constructed, providing a precise topographical model of the moons surface.
These DEMs are prone to voids containing no data, making the map less reliable for scientific purposes and future moon missions.
This paper uses a machine learning model to allow for the technique of image inpainting to be used with lunar DEMs.
Image inpainting uses pixel data from an image to generate missing data to fill a void.
DEMs can be thought of mathematically as identical to a single channel (greyscale) image, a two-dimensional array with "pixel" values corresponding to height, and so the technique of image inpainting can be easily applied to DEMs.
A Generative Adversarial Network (GAN) based on a fully convolutional architecture was used for the inpainting.

\end{abstract}

\section{Introduction}
\label{sec:org2f57e2f}

Requirements:
\begin{itemize}
\item Explanation of the problem
\item Background context
\item Objectives of the project
\end{itemize}

Sensory data from the Lunar Orbiter Laser Altimeter (LOLA) has been used for the construction of Lunar digital elevation models (DEMs) since its launch upon the Lunar Reconnaissance Orbiter in 2009.
LOLA has several primary objectives as part of a series of robotic missions that aim to pave the way for a permanent human presence on the Moon, including determining the global topography of the lunar surface at meter-scale resolution.
Topographical data from LOLA will be used to facilitate the selection of future landing sites, so accuracy and completeness are essential.
\autocite{chinLunarReconnaissanceOrbiter2007}

The LRO is in a polar orbit around the moon, with LOLA scanning the surface in swathes 50 to 60m wide.
These swathes have an average separation ranging between order 1.2km and order 200m\autocite{smithLunarOrbiterLaser2010}.
To construct a complete DEM from these swathes, they must be combined, which often results in artefacts from the blending method used.
LOLA DEMs are also prone to no-data voids from topographical features such as craters blocking the reflection of the laser depending on the angle at which LOLA is scanning from.

\printbibliography
\end{document}